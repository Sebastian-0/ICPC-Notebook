\documentclass[8pt,a4paper,landscape,oneside]{amsart}
\usepackage{amsmath, amsthm, amssymb, amsfonts}
\usepackage[T1]{fontenc}
\usepackage[utf8]{inputenc}
\usepackage{multicol}
\usepackage{fancyhdr}

%\usepackage{booktabs}
%\usepackage{caption}
%\usepackage{color}
%\usepackage{float}
\usepackage{fullpage}
%\usepackage{subcaption}
%\usepackage[scaled]{beramono}
\usepackage{titling}
%\usepackage{datetime}
%\usepackage{enumitem}
%\usepackage{bm}

% Minted
\usepackage{minted}
\newcommand{\code}[1]{\inputminted[fontsize=\normalsize,baselinestretch=1]{java}{code/#1}}

\newcommand{\bigO}{\mathcal{O}}

\pagestyle{fancy}
\lhead{Flip the tables - Lund University}
\rhead{\thepage}
\cfoot{}
\setlength{\headheight}{15.2pt}
\setlength{\droptitle}{-20pt}
\posttitle{\par\end{center}}
\renewcommand{\headrulewidth}{0.4pt}
\renewcommand{\footrulewidth}{0.4pt}

\begin{document}

\begin{multicols*}{3}
%\maketitle
\thispagestyle{fancy}
\vspace{-3em}

\tableofcontents
\section{Code Templates}
  \subsection{KattIO}
  \code{Kattio.java}
  
\section{Data Structures}
  \subsection{Binary Search}
  Time complexity is $\bigO(n \log(n))$ and space complexity is $\bigO(1)$.
  \code{Structures/BinarySearch.java}
  \subsection{Sorting}
  Time complexity is $\bigO(n \log(n))$ for both algorithms and space complexity is $\bigO(\log(n))$.
  \begin{itemize}
  \item \texttt{Collections.sort()} uses Merge Sort
  \item \texttt{Arrays.sort()} uses Quick Sort
  \end{itemize}
  
  \subsection{Fenwick tree}
  Time complexity is $\bigO(\log(n))$ for all operations and space complexity is $\bigO(1)$.
  \code{Structures/Fenwick.java}
  
\section{Graph Algorithms}
  \subsection{Dijsktra's Algorithm}
  Time complexity is $\bigO(|E| \log{|V|})$ and space complexity is $\bigO(|V|)$.
  \code{Graphs/Dijkstras.java}
  \subsection{Prim's Algorithm}
  Time complexity is $\bigO(|E| \log{|V|})$ and space complexity is $\bigO(|V|)$. If you want to have the resulting tree, add a parent node and a list of child nodes to all nodes. Set the parent when you update the cost of another node, and add yourself to your parent's list of children when you are popped.
  \code{Graphs/MST.java}
        
\section{Acieving AC on a solved problem}
    \subsection{WA}
        \begin{itemize}
        \item Check that minimal input passes.
        \item Can an int overflow?
        \item Reread the problem statement.
        \item Start creating small testcases with python.
        \item Does cout print with high enough precision?
        \item Abstract the implementation.
        \end{itemize}
    \subsection{TLE}
        \begin{itemize}
        \item Is the solution sanity checked?
        \item Use pypy instead of python.
        \item Rewrite in C++ or Java.
        \item Can we apply DP anywhere?
        \item To minimize penalty time you should create a worst case input (if easy) to test on.
        \end{itemize}
    \subsection{RTE}
        \begin{itemize}
        \item Recursion limit in python?
        \item Arrayindex out of bounds?
        \item Division by $0$?
        \item Modifying iterator while iterating over it?
        \item Not using a well definied operator for Collections.sort?
        \item If nothing makes sense and the end of the contest is approaching you 
            can binary search over where the error is with try-except.
        \end{itemize}
    \subsection{MLE}
        \begin{itemize}
        \item Create objects outside recursive function.
        \item Rewrite recursive solution to itterative with an own stack.
        \end{itemize}



\end{multicols*}
\end{document}
